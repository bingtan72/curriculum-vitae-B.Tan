%----------------------------------------------------------------------------------------
% Dr. Ke Li's CV LaTex Template (version 1.0)
%
% This template has been downloaded and developed from:
% http://www.LaTeXTemplates.com
%
% Original author:
% Xavier Danaux (xdanaux@gmail.com)
%
% License:
% CC BY-NC-SA 3.0 (http://creativecommons.org/licenses/by-nc-sa/3.0/)
%
% Important note:
% This template requires the moderncv.cls and .sty files to be in the same
% directory as this .tex file. These files provide the resume style and themes
% used for structuring the document.
%
%----------------------------------------------------------------------------------------

%----------------------------------------------------------------------------------------
%	PACKAGES AND OTHER DOCUMENT CONFIGURATIONS
%----------------------------------------------------------------------------------------

\documentclass[11pt,a4paper,sans]{moderncv} % Font sizes: 10, 11, or 12; paper sizes: a4paper, letterpaper, a5paper, legalpaper, executivepaper or landscape; font families: sans or roman
\moderncvstyle{classic} % CV theme - options include: 'casual' (default), 'classic', 'oldstyle' and 'banking'
\moderncvcolor{blue} % CV color - options include: 'blue' (default), 'orange', 'green', 'red', 'purple', 'grey' and 'black'

\usepackage{lipsum} % Used for inserting dummy 'Lorem ipsum' text into the template

\usepackage[scale=0.85]{geometry} % Reduce document margins

\usepackage{footmisc}
\renewcommand{\thefootnote}{}

%----------------------------------------------------------------------------------------
%	NAME AND CONTACT INFORMATION SECTION
%----------------------------------------------------------------------------------------

\firstname{Bing} % Your first name
\familyname{Tan} % Your last name

% All information in this block is optional, comment out any lines you don't need
\title{Master student}
\address{Institute of Fundamental and Frontier Sciences\\ University of Electronic Science and Technology of China\\}{Chengdu, China}
%\mobile{+(44) 790-790-8206}
\email{bingtan72@gmail.com}
\homepage{bingtan72.github.io/}{http://bingtan72.github.io/} % The first argument is the url for the clickable link, the second argument is the url displayed in the template - this allows special characters to be displayed such as the tilde in this example
% \extrainfo{additional information}
\photo[70pt][0.4pt]{pictures/bingtan.jpg} % The first bracket is the picture height, the second is the thickness of the frame around the picture (0pt for no frame)
%\quote{"A witty and playful quotation" - John Smith}
%\quote{\footnotesize Last updated: \today}

% Colors
\usepackage{xcolor}	 % Required for custom colors
% Define a few colors for making text stand out within the presentation
\definecolor{mygreen}{RGB}{44,85,17}
\definecolor{DarkGreen}{RGB}{0.000000,0.392157,0.000000}
\definecolor{myblue}{RGB}{34,31,217}
\definecolor{mybrown}{RGB}{194,164,113}
\definecolor{myred}{RGB}{255,66,56}
% Use these colors within the presentation by enclosing text in the commands below
\newcommand*{\mygreen}[1]{\textcolor{mygreen}{#1}}
\newcommand*{\mydgreen}[1]{\textcolor{DarkGreen}{#1}}
\newcommand*{\myblue}[1]{\textcolor{myblue}{#1}}
\newcommand*{\mybrown}[1]{\textcolor{mybrown}{#1}}
\newcommand*{\myred}[1]{\textcolor{myred}{#1}}

%----------------------------------------------------------------------------------------

\begin{document}

\makecvtitle % Print the CV title

%----------------------------------------------------------------------------------------
%	RESEARCH INTEREST SECTION
%----------------------------------------------------------------------------------------

\section{Research Interests}
\cvitem{}{Optimization algorithms, theory, applications}
\cvitem{}{Variational inequality}
\cvitem{}{Image Processing}

\section{Education}
\cventry{2018 -- 2021}{Masters of Mathematic}{Institute of Fundamental and Frontier Sciences}{University of Electronic Science and Technology of China}{China}{Supervisor: Prof. \href{https://publons.com/researcher/2535950/songxiao-li/}{Songxiao Li}  and Prof. \href{https://www.scopus.com/authid/detail.uri?authorId=14023330000}{Xiaolong Qin}}
\cventry{2014 -- 2018}{Bachelor of Mathematic}{School of Science}{Southwest Petroleum University}{China}{}

\section{Publications}

\subsection{Journal papers}
\cvitem{JAAC}{\textbf{Bing Tan}, Zheng Zhou, {Xiaolong Qin}*. Accelerated projection-based forward-backward splitting algorithms for monotone inclusion problems. \textit{J. Appl. Anal. Comput.} 2020, in press.}

\cvitem{JAAC}{{Zheng Zhou}*, \textbf{Bing Tan}, Songxiao Li. An inertial shrinking projection algorithm for split common fixed point problems. \textit{J. Appl. Anal. Comput.} 2020, in press.}

\cvitem{Mathematics}{\textbf{Bing Tan}, Shanshan Xu, {Songxiao Li}*. Modified inertial hybrid and shrinking projection algorithms for solving fixed point problems. \textit{Mathematics} 2020, 8(2), 236.  }

\cvitem{Mathematics}{Yinglin Luo, {Meijuan Shang}*, \textbf{Bing Tan}. A general inertial viscosity type method for nonexpansive mappings and its applications in signal processing. \textit{Mathematics} 2020, 8(2), 288.}

\cvitem{JNCA}{\textbf{Bing Tan}, Shanshan Xu, {Songxiao Li}*. Inertial shrinking projection algorithms for solving hierarchical variational inequality problems. \textit{J. Nonlinear Convex Anal.} 2020, in press.}

\cvitem{JNCA}{Yinglin Luo, \textbf{Bing Tan}*, A self-adaptive inertial extragradient algorithm for solving pseudo-monotone variational inequality in Hilbert spaces. \textit{J. Nonlinear Convex Anal.} 2020, in press.}

\cvitem{JNCA}{{Liya Liu}*, \textbf{Bing Tan}, {Sun Young Cho}*. On the resolution of variational inequality problems with a double-hierarchical structure. Submitted to \textit{J. Nonlinear Convex Anal.} 2020, 21(2): 377--386.}
\subsection{Preprints}
\cvitem{NFAO}{Jingjing Fan, {Xiaolong Qin}*, \textbf{Bing Tan}. Convergence of an inertial shadow Douglas-Rachford splitting for monotone inclusions. Submitted to \textit{Numerical Functional Analysis and Optimization}.}


\cvitem{AA}{\textbf{Bing Tan}, Songxiao Li, {Xiaolong Qin}*. Strong convergence of inertial Mann algorithms for solving hierarchical fixed point problems. Submitted to \textit{Applicable Analysis}.}

\section{Awards}
\cvitem{2019}{First-class scholarship of University of Electronic Science and Technology of China.}
\cvitem{2018}{Second-class scholarship of University of Electronic Science and Technology of China.}

%\section{Professional Services}

%\subsection{Memberships}
%\cvitem{2010 -- 2014}{IEEE student member}
%\cvitem{2013 -- now}{ACM professional member}


\section{Computer skills}
\cvitem{}{MATLAB, \LaTeX, Microsoft Office.}


\footnote{Updated by \today}
\end{document}
